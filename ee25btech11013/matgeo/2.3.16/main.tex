\let\negmedspace\undefined
\let\negthickspace\undefined
\documentclass[journal]{IEEEtran}
\usepackage[a5paper, margin=10mm, onecolumn]{geometry}
%\usepackage{lmodern} % Ensure lmodern is loaded for pdflatex
\usepackage{tfrupee} % Include tfrupee package

\setlength{\headheight}{1cm} % Set the height of the header box
\setlength{\headsep}{0mm}     % Set the distance between the header box and the top of the text

\usepackage{gvv-book}
\usepackage{gvv}
\usepackage{cite}
\usepackage{amsmath,amssymb,amsfonts,amsthm}
\usepackage{algorithmic}
\usepackage{graphicx}
\usepackage{textcomp}
\usepackage{xcolor}
\usepackage{txfonts}
\usepackage{listings}
\usepackage{enumitem}
\usepackage{mathtools}
\usepackage{gensymb}
\usepackage{comment}
\usepackage[breaklinks=true]{hyperref}
\usepackage{tkz-euclide} 
\usepackage{listings}
% \usepackage{gvv}                                        
\def\inputGnumericTable{}                                 
\usepackage[latin1]{inputenc}                                
\usepackage{color}                                            
\usepackage{array}                                            
\usepackage{longtable}                                       
\usepackage{calc}                                             
\usepackage{multirow}                                         
\usepackage{hhline}                                           
\usepackage{ifthen}                                           
\usepackage{lscape}
\usepackage{circuitikz}
\tikzstyle{block} = [rectangle, draw, fill=blue!20, 
    text width=4em, text centered, rounded corners, minimum height=3em]
\tikzstyle{sum} = [draw, fill=blue!10, circle, minimum size=1cm, node distance=1.5cm]
\tikzstyle{input} = [coordinate]
\tikzstyle{output} = [coordinate]


\begin{document}

\bibliographystyle{IEEEtran}
\vspace{3cm}

\title{2.3.16}
\author{EE25BTECH11013 - Bhargav}
\maketitle
% \newpage
% \bigskip
{\let\newpage\relax\maketitle}

\renewcommand{\thefigure}{\theenumi}
\renewcommand{\thetable}{\theenumi}
\setlength{\intextsep}{10pt} % Space between text and floats


\numberwithin{equation}{enumi}
\numberwithin{figure}{enumi}
\renewcommand{\thetable}{\theenumi}

\textbf{Question}:\\
If $\vec{p}$ is a unit vector and $(\vec{x}-\vec{p})\cdot(\vec{x}+\vec{p})=80$, then find $\norm{\vec{x}}$. \\
\solution \\
\begin{align}
(\vec{x}-\vec{p})^\top (\vec{x}+\vec{p})
\end{align}
\begin{align}
= \vec{x}^\top(\vec{x}+\vec{p}) - \vec{p}^\top(\vec{x}+\vec{p})
\end{align}
\begin{align}
= \vec{x}^\top \vec{x} + \vec{x}^\top \vec{p} - \vec{p}^\top \vec{x} - \vec{p}^\top \vec{p}.
\end{align}


Since $\vec{x}^\top \vec{p} = \vec{p}^\top \vec{x}$, the mixed terms cancel:

\begin{align}
 = (\vec{x}-\vec{p})^\top (\vec{x}+\vec{p})
\end{align}
\begin{align}
= \norm{\vec{x}}^{2} - \norm{\vec{p}}^{2}
\end{align}

Given $\vec{p}$ is a unit vector, $\|\vec{p}\|^2 = 1$. Also, it is given that

\begin{align}
(\vec{x}-\vec{p})^\top (\vec{x}+\vec{p}) = 80.
\end{align}

Thus,

\begin{align}
\norm{\vec{x}}^2 - 1 = 80 
\end{align}

\begin{align}
\norm{\vec{x}}^2 = 81
\end{align}

\begin{align}
\norm{\vec{x}} = 9.
\end{align}

 The theoretical solution can be verified by example.
    
    \vfill
    
    Assume that $\vec{p}$ is the unit vector $\myvec{1 \\ 0}$.
    
    \vfill

    Then from the code we get a possible vector $\vec{x}$ would be $\myvec{9 \\ 0}$.
    
    \vfill
    
    The magnitude of the $\vec{x}$ is verified to be $9$.

\end{document}


