\documentclass{beamer}
\let\vec\mathbf
\mode<presentation>
\usepackage{amsmath}
\usepackage{amssymb}
%\usepackage{advdate}
\usepackage{adjustbox}
%\usepackage{subcaption}
\usepackage{enumitem}
\usepackage{multicol}
\usepackage{mathtools}
\usepackage{listings}
\usepackage{url}
\usetheme{Boadilla}
\usecolortheme{lily}
\setbeamertemplate{footline}
{
  \leavevmode%
  \hbox{%
  \begin{beamercolorbox}[wd=\paperwidth,ht=2.25ex,dp=1ex,right]{author in head/foot}%
    \insertframenumber{} / \inserttotalframenumber\hspace*{2ex} 
  \end{beamercolorbox}}%
  \vskip0pt%
}
\setbeamertemplate{navigation symbols}{}
\providecommand{\nCr}[2]{\,^{#1}C_{#2}} % nCr
\providecommand{\nPr}[2]{\,^{#1}P_{#2}} % nPr
\providecommand{\mbf}{\mathbf}
\providecommand{\pr}[1]{\ensuremath{\Pr\left(#1\right)}}
\providecommand{\qfunc}[1]{\ensuremath{Q\left(#1\right)}}
\providecommand{\sbrak}[1]{\ensuremath{{}\left[#1\right]}}
\providecommand{\lsbrak}[1]{\ensuremath{{}\left[#1\right.}}
\providecommand{\rsbrak}[1]{\ensuremath{{}\left.#1\right]}}
\providecommand{\brak}[1]{\ensuremath{\left(#1\right)}}
\providecommand{\lbrak}[1]{\ensuremath{\left(#1\right.}}
\providecommand{\rbrak}[1]{\ensuremath{\left.#1\right)}}
\providecommand{\cbrak}[1]{\ensuremath{\left\{#1\right\}}}
\providecommand{\lcbrak}[1]{\ensuremath{\left\{#1\right.}}
\providecommand{\rcbrak}[1]{\ensuremath{\left.#1\right\}}}
\theoremstyle{remark}
\newtheorem{rem}{Remark}
\newcommand{\sgn}{\mathop{\mathrm{sgn}}}

\providecommand{\res}[1]{\Res\displaylimits_{#1}} 
\providecommand{\norm}[1]{\left\lVert#1\right\rVert}
\providecommand{\mtx}[1]{\mathbf{#1}}
\providecommand{\abs}[1]{\left\vert#1\right\vert}
\providecommand{\fourier}{\overset{\mathcal{F}}{ \rightleftharpoons}}
%\providecommand{\hilbert}{\overset{\mathcal{H}}{ \rightleftharpoons}}
\providecommand{\system}{\overset{\mathcal{H}}{ \longleftrightarrow}}
	%\newcommand{\solution}[2]{\textbf{Solution:}{#1}}
%\newcommand{\solution}{\noindent \textbf{Solution: }}align
\providecommand{\dec}[2]{\ensuremath{\overset{#1}{\underset{#2}{\gtrless}}}}
\newcommand{\myvec}[1]{\ensuremath{\begin{pmatrix}#1\end{pmatrix}}}

\title{Matrices in Geometry - 2.8.20}
\author{EE25BTECH11037  Divyansh}
\date{Sept, 2025}

\begin{document}

\maketitle


\section{Problem}
\begin{frame}
\frametitle{Problem Statement}
If $\abs{\vec{a}+\vec{b}}=\abs{\vec{a}-\vec{b}}$, then prove that $\vec{a}$ and $\vec{b}$ are orthogonal.
\end{frame}

\section{Solution}
\begin{frame}{Solution}
   \textbf{Given: } 
 $\vec{A}\myvec{2\\-2}$, $\vec{B}\myvec{7\\3}$, $\vec{C}\myvec{11 \\ -1}$ and $\vec{D}\myvec{6 \\ -6}$.
\begin{align}
    \norm{\vec{a}+\vec{b}}=\norm{\vec{a}-\vec{b}}\\
    \text{Two vectors $\vec{a}$ and $\vec{b}$ are orthogonal if,} \ \ \ \vec{a}^{\top}\vec{b}=0\\
    \norm{\vec{a}+\vec{b}}^2=\norm{\vec{a}-\vec{b}}^2
\end{align}
\end{frame}

\begin{frame}{Solution}
\text{We know that $\norm{\vec{a}}^2=\vec{a}^{\top}\vec{a}$}
\begin{align}
    \implies \brak{\vec{a}+\vec{b}}^{\top}\brak{\vec{a}+\vec{b}}= \brak{\vec{a}-\vec{b}}^{\top}\brak{\vec{a}-\vec{b}}\\
    \implies \vec{a}^{\top}\vec{a} + 2\vec{a}^{\top}\vec{b}+ \vec{b}^{\top}\vec{b}=\vec{a}^{\top}\vec{a} - 2\vec{a}^{\top}\vec{b}+ \vec{b}^{\top}\vec{b}\\
    \implies 4\vec{a}^{\top}\vec{b}=0\\
    \implies \vec{a}^{\top}\vec{b}=0
\end{align}
This shows that $\vec{a}$ and $\vec{b}$ are orthogonal.\\
\end{frame}
\begin{frame}{Solution}
Let us try to show this for an example\\
\begin{align}
\text{Let } \vec{a}=\myvec{1 \\ -1} \text{ and } \vec{b}=\myvec{1 \\ 1}\\
\brak{\vec{a+b}}=\myvec{2 \\ 0}, \brak{\vec{a-b}}=\myvec{0\\-2}\\
\text{We can clearly see that  $\norm{\vec{a}+\vec{b}}=\norm{\vec{a}-\vec{b}}=2$ }\\
\vec{a}^{\top}\vec{b}=\myvec{1 & -1}\myvec{1 \\ 1}=1-1 = 0
\end{align}
This property is also proved for an example.\\
Hence, Proved    
\end{frame}


\end{document}